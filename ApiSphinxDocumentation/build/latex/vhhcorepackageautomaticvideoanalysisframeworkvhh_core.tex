%% Generated by Sphinx.
\def\sphinxdocclass{report}
\documentclass[letterpaper,10pt,english,openany,oneside]{sphinxmanual}
\ifdefined\pdfpxdimen
   \let\sphinxpxdimen\pdfpxdimen\else\newdimen\sphinxpxdimen
\fi \sphinxpxdimen=.75bp\relax

\PassOptionsToPackage{warn}{textcomp}
\usepackage[utf8]{inputenc}
\ifdefined\DeclareUnicodeCharacter
% support both utf8 and utf8x syntaxes
  \ifdefined\DeclareUnicodeCharacterAsOptional
    \def\sphinxDUC#1{\DeclareUnicodeCharacter{"#1}}
  \else
    \let\sphinxDUC\DeclareUnicodeCharacter
  \fi
  \sphinxDUC{00A0}{\nobreakspace}
  \sphinxDUC{2500}{\sphinxunichar{2500}}
  \sphinxDUC{2502}{\sphinxunichar{2502}}
  \sphinxDUC{2514}{\sphinxunichar{2514}}
  \sphinxDUC{251C}{\sphinxunichar{251C}}
  \sphinxDUC{2572}{\textbackslash}
\fi
\usepackage{cmap}
\usepackage[T1]{fontenc}
\usepackage{amsmath,amssymb,amstext}
\usepackage{babel}



\usepackage{times}
\expandafter\ifx\csname T@LGR\endcsname\relax
\else
% LGR was declared as font encoding
  \substitutefont{LGR}{\rmdefault}{cmr}
  \substitutefont{LGR}{\sfdefault}{cmss}
  \substitutefont{LGR}{\ttdefault}{cmtt}
\fi
\expandafter\ifx\csname T@X2\endcsname\relax
  \expandafter\ifx\csname T@T2A\endcsname\relax
  \else
  % T2A was declared as font encoding
    \substitutefont{T2A}{\rmdefault}{cmr}
    \substitutefont{T2A}{\sfdefault}{cmss}
    \substitutefont{T2A}{\ttdefault}{cmtt}
  \fi
\else
% X2 was declared as font encoding
  \substitutefont{X2}{\rmdefault}{cmr}
  \substitutefont{X2}{\sfdefault}{cmss}
  \substitutefont{X2}{\ttdefault}{cmtt}
\fi


\usepackage[Bjarne]{fncychap}
\usepackage{sphinx}

\fvset{fontsize=\small}
\usepackage{geometry}


% Include hyperref last.
\usepackage{hyperref}
% Fix anchor placement for figures with captions.
\usepackage{hypcap}% it must be loaded after hyperref.
% Set up styles of URL: it should be placed after hyperref.
\urlstyle{same}

\usepackage{sphinxmessages}
\setcounter{tocdepth}{3}
\setcounter{secnumdepth}{3}


\title{VHH Core Package: Automatic Video Analysis Framework (vhh\_core)}
\date{Jun 09, 2020}
\release{1.0.0}
\author{Daniel Helm}
\newcommand{\sphinxlogo}{\vbox{}}
\renewcommand{\releasename}{Release}
\makeindex
\begin{document}

\pagestyle{empty}
\sphinxmaketitle
\pagestyle{plain}
\sphinxtableofcontents
\pagestyle{normal}
\phantomsection\label{\detokenize{index::doc}}


Detecting cinematographic techniques such as Shot Boundary Detection (SBD), Shot Type Classification (STC) and Camera
Movements Classification (CMC) is a fundamental task for automatic film archival. Therefore, a software framework is
developed to process historical films related to the time of the liberation phase of Nazi concentration camps during
the Second World War {[}\sphinxhref{https://www.vhh-project.eu/}{Visual History of the Holocaust} %
\begin{footnote}[1]\sphinxAtStartFootnote
\sphinxurl{https://www.vhh-project.eu/}
%
\end{footnote} (VHH){]}. This framework is able to detect and classify SBDs
(\sphinxhref{https://github.com/dahe-cvl/vhh\_sbd}{vhh\_sbd} %
\begin{footnote}[3]\sphinxAtStartFootnote
\sphinxurl{https://github.com/dahe-cvl/vhh\_sbd}
%
\end{footnote}), STCs (\sphinxhref{https://github.com/dahe-cvl/vhh\_stc}{vhh\_stc} %
\begin{footnote}[4]\sphinxAtStartFootnote
\sphinxurl{https://github.com/dahe-cvl/vhh\_stc}
%
\end{footnote}) and CMCs (\sphinxhref{https://github.com/dahe-cvl/vhh\_cmc}{vhh\_cmc} %
\begin{footnote}[5]\sphinxAtStartFootnote
\sphinxurl{https://github.com/dahe-cvl/vhh\_cmc}
%
\end{footnote}) by using deep learning\sphinxhyphen{}based as well as optical flow\sphinxhyphen{}based
approaches {[}XX{]}{[}XX{]}{[}XX{]}. This documentation is separated into the following sections: In Section \sphinxstyleemphasis{System Overview} and
\sphinxstyleemphasis{Process Pipeline} an overview of the code structure as well as the automatic annotation process pipeline is demonstrated.
Furthermore, the package structure and the setup and usage description is visualized in Section \sphinxstyleemphasis{Package Structure} and
\sphinxstyleemphasis{Setup Instructions}. Finally, this documentation closes with an API description of all classes, modules and their
members in Section \sphinxstyleemphasis{API Description}.


\chapter{System Overview}
\label{\detokenize{index:system-overview}}
The software architecture is visualized in the below figure. The \sphinxhref{https://github.com/dahe-cvl/vhh\_core}{vhh\_core} %
\begin{footnote}[2]\sphinxAtStartFootnote
\sphinxurl{https://github.com/dahe-cvl/vhh\_core}
%
\end{footnote} is divided in multiple classes. The core
class is called \sphinxstyleemphasis{MainController} and administers the external plugin modules for automatic shot analysis in historical
films: \sphinxhref{https://github.com/dahe-cvl/vhh\_sbd}{vhh\_sbd} \sphinxfootnotemark[3] (SBD), \sphinxhref{https://github.com/dahe-cvl/vhh\_stc}{vhh\_stc} \sphinxfootnotemark[4] (STC) and \sphinxhref{https://github.com/dahe-cvl/vhh\_cmc}{vhh\_cmc} \sphinxfootnotemark[5] (CMC). Moreover, central functionality such as the communication
with the VHH\sphinxhyphen{}MMSI system is handled in this class. The class \sphinxstyleemphasis{VhhRestApi} includes methods to communicate with the
VHH\sphinxhyphen{}MMSI database via RestAPI endpoints. A various number of parameters defined and set in the \sphinxstyleemphasis{config\_core.yaml}
file is managed by the class \sphinxstyleemphasis{Configuration}. Furthermore, each shot analysis plugin can be configured with separate
configuration files stored in the \sphinxstyleemphasis{./config} folder in this repositories.

\begin{figure}[htbp]
\centering

\noindent\sphinxincludegraphics{{software_architecture_core}.pdf}
\end{figure}


\chapter{Process Pipeline}
\label{\detokenize{index:process-pipeline}}
\begin{figure}[htbp]
\centering

\noindent\sphinxincludegraphics{{process_pipeline_core}.pdf}
\end{figure}


\chapter{Package Overview}
\label{\detokenize{index:package-overview}}
The following list gives an overview of the folder structure of this python repository:

\sphinxstyleemphasis{name of repository}: vhh\_core
\begin{itemize}
\item {} 
\sphinxstylestrong{ApiSphinxDocumentation/}: includes all files to generate the documentation as well as the created documentations (html, pdf). The results are stored in the build folder (e.g. xx/build/latex/vhh\_core.pdf)

\item {} 
\sphinxstylestrong{config/}: This folder includes the required configuration file. For each plugin (sbd, stc, cmc, …) there is one subdirectory holding the corresponding configuration file.

\item {} 
\sphinxstylestrong{Demo/}: This folder includes a all scripts to run this application. Furthermore, scripts to setup the folder structure (video and results storage) as well as to setup the environment for running this application.

\item {} 
\sphinxstylestrong{Develop/}: This folder includes additional scripts used during developing phase of this application. (e.g. the document\_builder script to generate the package documentation).

\item {} 
\sphinxstylestrong{README.md}: This file gives a brief description of this repository (e.g. link to this documentation)

\item {} 
\sphinxstylestrong{requirements.txt}: this file holds all python dependencies and is needed to install the package in your own virtual environment

\item {} 
\sphinxstylestrong{setup.py}: this script is needed to install the stc package in your own virtual environment

\end{itemize}


\chapter{Setup  instructions}
\label{\detokenize{index:setup-instructions}}
This package includes a setup.py script and a requirements.txt file which are needed to install this package for custom applications.
The following instructions have to be done to used this library in your own application:

\sphinxstylestrong{Requirements:}
\begin{itemize}
\item {} 
Ubuntu 18.04 LTS

\item {} 
CUDA 10.1 + cuDNN

\item {} 
python version 3.6.x

\end{itemize}

\sphinxstylestrong{Create a virtual environment:}
\begin{itemize}
\item {} 
create a folder to a specified path (e.g. /xxx/vhh\_core/)

\item {} 
python3 \sphinxhyphen{}m venv /xxx/vhh\_core/

\end{itemize}

\sphinxstylestrong{Activate the environment:}
\begin{itemize}
\item {} 
source /xxx/vhh\_core/bin/activate

\end{itemize}

\sphinxstylestrong{Checkout vhh\_core repository to a specified folder:}
\begin{itemize}
\item {} 
git clone \sphinxurl{https://github.com/dahe-cvl/vhh\_core}

\end{itemize}

\sphinxstylestrong{Install the stc package and all dependencies:}
\begin{itemize}
\item {} 
change to the root directory of the repository (includes setup.py)

\item {} 
python setup.py install

\end{itemize}

\begin{sphinxadmonition}{note}{Note:}
You can check the success of the installation by using the commend \sphinxstyleemphasis{pip list}. This command should give you a list with all installed python packages and it should include \sphinxstyleemphasis{vhh\_core}
\end{sphinxadmonition}

\begin{sphinxadmonition}{note}{Note:}
Currently there is an issue in the \sphinxstyleemphasis{setup.py} script. Therefore the pytorch libraries have to be installed manually by running the following command:
\sphinxstyleemphasis{pip install torch==1.5.0+cu101 torchvision==0.6.0+cu101 \sphinxhyphen{}f https://download.pytorch.org/whl/torch\_stable.html}
\end{sphinxadmonition}


\chapter{Parameter Description}
\label{\detokenize{index:parameter-description}}
DEBUG\_FLAG
This parameter is used to activate or deactivate the debug mode.



PEM\_PATH
This parameter specifies the path tp the certificate file needed to provide vhh\sphinxhyphen{}mmsi api access.



ROOT\_URL
This parameter specifies the root url for the RestApi endpoints.



VIDEO\_DOWNLOAD\_PATH
This parameter specifies the video download path.



CLEANUP\_FLAG
This parameter is used to activate/deactivate the clean\sphinxhyphen{}up mode. The clean\sphinxhyphen{}up mode deletes all downloaded videos and the corresponding generated results.



RESULTS\_ROOT\_DIR
This parameter specifies the results path for each individual plugin.



SBD\_CONFIG\_FILE
This parameter specifies the configuration path (.yaml) for the shot boundary detection plugin module



STC\_CONFIG\_FILE
This parameter specifies the configuration path (.yaml) for the shot type classification plugin module



CMC\_CONFIG\_FILE
This parameter specifies the configuration path (.yaml) for the camera movement classification module




\chapter{API Description}
\label{\detokenize{index:api-description}}
This section gives an overview of all classes and modules in \sphinxstyleemphasis{vhh\_core} as well as a brief configuration parameter description.


\section{Configuration class}
\label{\detokenize{Configuration:configuration-class}}\label{\detokenize{Configuration::doc}}\index{Configuration (class in Configuration)@\spxentry{Configuration}\spxextra{class in Configuration}}

\begin{fulllineitems}
\phantomsection\label{\detokenize{Configuration:Configuration.Configuration}}\pysiglinewithargsret{\sphinxbfcode{\sphinxupquote{class }}\sphinxcode{\sphinxupquote{Configuration.}}\sphinxbfcode{\sphinxupquote{Configuration}}}{\emph{\DUrole{n}{config\_file}\DUrole{p}{:} \DUrole{n}{str}}}{}
Bases: \sphinxcode{\sphinxupquote{object}}

This class is needed to read the configuration parameters specified in the configuration.yaml file.
The instance of the class is holding all parameters during runtime.

\begin{sphinxadmonition}{note}{Note:}
e.g. ./config/CORE/config.yaml
\begin{quote}

the yaml file is separated in multiple sections
config{[}‘Development’{]}
config{[}‘Security’{]}
config{[}‘VhhCore’{]}
config{[}‘ApiEndpoints’{]}
config{[}‘PluginConfigs’{]}

whereas each section should hold related and meaningful parameters.
\end{quote}
\end{sphinxadmonition}
\index{loadConfig() (Configuration.Configuration method)@\spxentry{loadConfig()}\spxextra{Configuration.Configuration method}}

\begin{fulllineitems}
\phantomsection\label{\detokenize{Configuration:Configuration.Configuration.loadConfig}}\pysiglinewithargsret{\sphinxbfcode{\sphinxupquote{loadConfig}}}{}{}
Method to load configurables from the specified configuration file

\end{fulllineitems}


\end{fulllineitems}



\section{Sbd class}
\label{\detokenize{Sbd:sbd-class}}\label{\detokenize{Sbd::doc}}\index{Sbd (class in Sbd)@\spxentry{Sbd}\spxextra{class in Sbd}}

\begin{fulllineitems}
\phantomsection\label{\detokenize{Sbd:Sbd.Sbd}}\pysiglinewithargsret{\sphinxbfcode{\sphinxupquote{class }}\sphinxcode{\sphinxupquote{Sbd.}}\sphinxbfcode{\sphinxupquote{Sbd}}}{\emph{\DUrole{n}{config}\DUrole{o}{=}\DUrole{default_value}{None}}}{}
Bases: \sphinxcode{\sphinxupquote{object}}

This class includes the interfaces and methods to use the plugin package SBD.
\index{run() (Sbd.Sbd method)@\spxentry{run()}\spxextra{Sbd.Sbd method}}

\begin{fulllineitems}
\phantomsection\label{\detokenize{Sbd:Sbd.Sbd.run}}\pysiglinewithargsret{\sphinxbfcode{\sphinxupquote{run}}}{\emph{\DUrole{n}{video\_instance\_list}\DUrole{o}{=}\DUrole{default_value}{None}}}{}
This method is used to run the shot boundary detection task
\begin{quote}\begin{description}
\item[{Parameters}] \leavevmode
\sphinxstyleliteralstrong{\sphinxupquote{video\_instance\_list}} \textendash{} parameter must hold a list of video objects (Class\sphinxhyphen{}type: Video)

\end{description}\end{quote}

\end{fulllineitems}


\end{fulllineitems}



\section{Stc class}
\label{\detokenize{Stc:stc-class}}\label{\detokenize{Stc::doc}}\index{Stc (class in Stc)@\spxentry{Stc}\spxextra{class in Stc}}

\begin{fulllineitems}
\phantomsection\label{\detokenize{Stc:Stc.Stc}}\pysiglinewithargsret{\sphinxbfcode{\sphinxupquote{class }}\sphinxcode{\sphinxupquote{Stc.}}\sphinxbfcode{\sphinxupquote{Stc}}}{\emph{\DUrole{n}{config}\DUrole{o}{=}\DUrole{default_value}{None}}}{}
Bases: \sphinxcode{\sphinxupquote{object}}

This class includes the interfaces and methods to use the plugin package STC.
\index{run() (Stc.Stc method)@\spxentry{run()}\spxextra{Stc.Stc method}}

\begin{fulllineitems}
\phantomsection\label{\detokenize{Stc:Stc.Stc.run}}\pysiglinewithargsret{\sphinxbfcode{\sphinxupquote{run}}}{}{}
This method is used to run the shot type classification task.

\end{fulllineitems}


\end{fulllineitems}



\section{Video class}
\label{\detokenize{Video:video-class}}\label{\detokenize{Video::doc}}\index{Video (class in Video)@\spxentry{Video}\spxextra{class in Video}}

\begin{fulllineitems}
\phantomsection\label{\detokenize{Video:Video.Video}}\pysiglinewithargsret{\sphinxbfcode{\sphinxupquote{class }}\sphinxcode{\sphinxupquote{Video.}}\sphinxbfcode{\sphinxupquote{Video}}}{\emph{\DUrole{n}{config}\DUrole{o}{=}\DUrole{default_value}{None}}}{}
Bases: \sphinxcode{\sphinxupquote{object}}

This class represents a video object.
\index{cleanup() (Video.Video method)@\spxentry{cleanup()}\spxextra{Video.Video method}}

\begin{fulllineitems}
\phantomsection\label{\detokenize{Video:Video.Video.cleanup}}\pysiglinewithargsret{\sphinxbfcode{\sphinxupquote{cleanup}}}{}{}
This method is used to cleanup all data related to the corresponding video ID. It deletes the generated results
of the sbd, stc and cmc plugin as well as the downloaded video file.

\end{fulllineitems}

\index{create\_video() (Video.Video method)@\spxentry{create\_video()}\spxextra{Video.Video method}}

\begin{fulllineitems}
\phantomsection\label{\detokenize{Video:Video.Video.create_video}}\pysiglinewithargsret{\sphinxbfcode{\sphinxupquote{create\_video}}}{\emph{\DUrole{n}{vid}}, \emph{\DUrole{n}{originalFileName}}, \emph{\DUrole{n}{url}}, \emph{\DUrole{n}{download\_path}}}{}
This method is used to fill all properties of a video.
\begin{quote}\begin{description}
\item[{Parameters}] \leavevmode\begin{itemize}
\item {} 
\sphinxstyleliteralstrong{\sphinxupquote{vid}} \textendash{} This parameter must hold a valid video id.

\item {} 
\sphinxstyleliteralstrong{\sphinxupquote{originalFileName}} \textendash{} This parameter must hold the original filename of a video

\item {} 
\sphinxstyleliteralstrong{\sphinxupquote{url}} \textendash{} This parameter must hold the download url of the video.

\item {} 
\sphinxstyleliteralstrong{\sphinxupquote{download\_path}} \textendash{} This parameter must hold the download path in the local storage.

\end{itemize}

\end{description}\end{quote}

\end{fulllineitems}

\index{download() (Video.Video method)@\spxentry{download()}\spxextra{Video.Video method}}

\begin{fulllineitems}
\phantomsection\label{\detokenize{Video:Video.Video.download}}\pysiglinewithargsret{\sphinxbfcode{\sphinxupquote{download}}}{\emph{\DUrole{n}{rest\_api\_instance}\DUrole{o}{=}\DUrole{default_value}{None}}}{}
This method is used to download the video into the local storage path.
\begin{quote}\begin{description}
\item[{Parameters}] \leavevmode
\sphinxstyleliteralstrong{\sphinxupquote{rest\_api\_instance}} \textendash{} This parameter must hold a valid VhhRestApi object.

\item[{Returns}] \leavevmode
This method returns the download status (true … successfully downloaded OR false … download failed).

\end{description}\end{quote}

\end{fulllineitems}

\index{is\_downloaded() (Video.Video method)@\spxentry{is\_downloaded()}\spxextra{Video.Video method}}

\begin{fulllineitems}
\phantomsection\label{\detokenize{Video:Video.Video.is_downloaded}}\pysiglinewithargsret{\sphinxbfcode{\sphinxupquote{is\_downloaded}}}{}{}
This method is used to check if a video is already downloaded.
\begin{quote}\begin{description}
\item[{Returns}] \leavevmode
This method returns a boolean flag (true … video already downloaded OR false … video does not exist).

\end{description}\end{quote}

\end{fulllineitems}

\index{printInfo() (Video.Video method)@\spxentry{printInfo()}\spxextra{Video.Video method}}

\begin{fulllineitems}
\phantomsection\label{\detokenize{Video:Video.Video.printInfo}}\pysiglinewithargsret{\sphinxbfcode{\sphinxupquote{printInfo}}}{}{}
This method summarizes all properties of this object and print it to the console.

\end{fulllineitems}


\end{fulllineitems}



\section{VhhRestApi class}
\label{\detokenize{VhhRestApi:vhhrestapi-class}}\label{\detokenize{VhhRestApi::doc}}\index{VhhRestApi (class in VhhRestApi)@\spxentry{VhhRestApi}\spxextra{class in VhhRestApi}}

\begin{fulllineitems}
\phantomsection\label{\detokenize{VhhRestApi:VhhRestApi.VhhRestApi}}\pysiglinewithargsret{\sphinxbfcode{\sphinxupquote{class }}\sphinxcode{\sphinxupquote{VhhRestApi.}}\sphinxbfcode{\sphinxupquote{VhhRestApi}}}{\emph{\DUrole{n}{config}\DUrole{o}{=}\DUrole{default_value}{None}}}{}
Bases: \sphinxcode{\sphinxupquote{object}}

This class includes the interfaces and methods to use the vhh restAPI interfaces provided by MaxRecall.
\index{downloadVideo() (VhhRestApi.VhhRestApi method)@\spxentry{downloadVideo()}\spxextra{VhhRestApi.VhhRestApi method}}

\begin{fulllineitems}
\phantomsection\label{\detokenize{VhhRestApi:VhhRestApi.VhhRestApi.downloadVideo}}\pysiglinewithargsret{\sphinxbfcode{\sphinxupquote{downloadVideo}}}{\emph{\DUrole{n}{url}}, \emph{\DUrole{n}{file\_name}}, \emph{\DUrole{n}{video\_format}}}{}
This method is used to download a video from the Vhh\sphinxhyphen{}MMSI system.
\begin{quote}\begin{description}
\item[{Parameters}] \leavevmode\begin{itemize}
\item {} 
\sphinxstyleliteralstrong{\sphinxupquote{url}} \textendash{} this parameter must hold a valid restApi endpoint.

\item {} 
\sphinxstyleliteralstrong{\sphinxupquote{file\_name}} \textendash{} This parameter represents the filename on the local storage.

\item {} 
\sphinxstyleliteralstrong{\sphinxupquote{video\_format}} \textendash{} This parameter must hold a valid video format extension (e.g. m4v)

\end{itemize}

\item[{Returns}] \leavevmode
This method returns a boolean flag which includes the state of the download process (true … sucessfully finished OR false … downlaod failed)

\end{description}\end{quote}

\end{fulllineitems}

\index{getAutomaticResults() (VhhRestApi.VhhRestApi method)@\spxentry{getAutomaticResults()}\spxextra{VhhRestApi.VhhRestApi method}}

\begin{fulllineitems}
\phantomsection\label{\detokenize{VhhRestApi:VhhRestApi.VhhRestApi.getAutomaticResults}}\pysiglinewithargsret{\sphinxbfcode{\sphinxupquote{getAutomaticResults}}}{\emph{\DUrole{n}{vid}}}{}
This method is used to get all automatic generated results from the VhhMMSI system.
\begin{quote}\begin{description}
\item[{Parameters}] \leavevmode
\sphinxstyleliteralstrong{\sphinxupquote{vid}} \textendash{} This parameter must hold a valid video identifier.

\item[{Returns}] \leavevmode
THis method returns the results (payload) as json format.

\end{description}\end{quote}

\end{fulllineitems}

\index{getListofVideos() (VhhRestApi.VhhRestApi method)@\spxentry{getListofVideos()}\spxextra{VhhRestApi.VhhRestApi method}}

\begin{fulllineitems}
\phantomsection\label{\detokenize{VhhRestApi:VhhRestApi.VhhRestApi.getListofVideos}}\pysiglinewithargsret{\sphinxbfcode{\sphinxupquote{getListofVideos}}}{}{}
This method is used to get a list of all available videos in the VHH\sphinxhyphen{}MMSI system.
\begin{quote}\begin{description}
\item[{Returns}] \leavevmode
This method returns a list of video objects (Class\sphinxhyphen{}type: Video) which holds all video specific meta\sphinxhyphen{}data.

\end{description}\end{quote}

\end{fulllineitems}

\index{getRequest() (VhhRestApi.VhhRestApi method)@\spxentry{getRequest()}\spxextra{VhhRestApi.VhhRestApi method}}

\begin{fulllineitems}
\phantomsection\label{\detokenize{VhhRestApi:VhhRestApi.VhhRestApi.getRequest}}\pysiglinewithargsret{\sphinxbfcode{\sphinxupquote{getRequest}}}{\emph{\DUrole{n}{url}}}{}
This method is used to send a get request to the Vhh\sphinxhyphen{}MMSI system.
\begin{quote}\begin{description}
\item[{Parameters}] \leavevmode
\sphinxstyleliteralstrong{\sphinxupquote{url}} \textendash{} this parameter must hold a valid restApi endpoint.

\item[{Returns}] \leavevmode
This method returns the original response including header as well as payload.

\end{description}\end{quote}

\end{fulllineitems}

\index{postAutomaticResults() (VhhRestApi.VhhRestApi method)@\spxentry{postAutomaticResults()}\spxextra{VhhRestApi.VhhRestApi method}}

\begin{fulllineitems}
\phantomsection\label{\detokenize{VhhRestApi:VhhRestApi.VhhRestApi.postAutomaticResults}}\pysiglinewithargsret{\sphinxbfcode{\sphinxupquote{postAutomaticResults}}}{\emph{\DUrole{n}{vid}}, \emph{\DUrole{n}{results\_np}}}{}
This method is used to post the automatic generated results to the VhhMMSI system.
\begin{quote}\begin{description}
\item[{Parameters}] \leavevmode\begin{itemize}
\item {} 
\sphinxstyleliteralstrong{\sphinxupquote{vid}} \textendash{} This parameter must hold a valid video identifier.

\item {} 
\sphinxstyleliteralstrong{\sphinxupquote{results\_np}} \textendash{} This parameter must hold a numpy array including the automatic generated results.

\end{itemize}

\end{description}\end{quote}

\end{fulllineitems}

\index{postRequest() (VhhRestApi.VhhRestApi method)@\spxentry{postRequest()}\spxextra{VhhRestApi.VhhRestApi method}}

\begin{fulllineitems}
\phantomsection\label{\detokenize{VhhRestApi:VhhRestApi.VhhRestApi.postRequest}}\pysiglinewithargsret{\sphinxbfcode{\sphinxupquote{postRequest}}}{\emph{\DUrole{n}{url}}, \emph{\DUrole{n}{data\_dict}}}{}
This method is used to send a post request to the Vhh\sphinxhyphen{}MMSI system.
\begin{quote}\begin{description}
\item[{Parameters}] \leavevmode\begin{itemize}
\item {} 
\sphinxstyleliteralstrong{\sphinxupquote{url}} \textendash{} this parameter must hold a valid restApi endpoint.

\item {} 
\sphinxstyleliteralstrong{\sphinxupquote{data\_dict}} \textendash{} this parameter must hold a valid list of dictionaries with the specified fields (see RestApi documentation).

\end{itemize}

\item[{Returns}] \leavevmode
This method returns the original response including header as well as payload

\end{description}\end{quote}

\end{fulllineitems}


\end{fulllineitems}



\section{MainController class}
\label{\detokenize{MainController:maincontroller-class}}\label{\detokenize{MainController::doc}}

\chapter{Indices and tables}
\label{\detokenize{index:indices-and-tables}}\begin{itemize}
\item {} 
\DUrole{xref,std,std-ref}{genindex}

\item {} 
\DUrole{xref,std,std-ref}{modindex}

\item {} 
\DUrole{xref,std,std-ref}{search}

\end{itemize}


\chapter{References}
\label{\detokenize{index:references}}


\renewcommand{\indexname}{Index}
\printindex
\end{document}